%% Erl�uterungen zu den Befehlen erfolgen unter
%% diesem Beispiel.                
\documentclass{scrartcl}
\usepackage[latin1]{inputenc}
\usepackage[T1]{fontenc}
\usepackage[ngerman]{babel}
\usepackage{amsmath}

\title{usbprog - Mikrocontroller Entwicklungskit}
\author{Benedikt Sauter}
\date{07. Februar 2007}
\begin{document}

\maketitle
\tableofcontents
\section{Was ist usbprog?}

Haupts�chlich ist usbprog eine universelle Hardware die auf 
der Computer Seite eine USB Schnittstelle hat und auf
dem Adapter eine SPI Schnittstelle
oder alternativ 4 Portleitungen und eine RS232 Schnittstelle.

�ber einen einfachen Update Mechanismus, kann man direkt
�ber die bestehende USB Verbindung die Software 
auf dem Adapter austauschen. Dies geht bequem �ber ein
grafisches Programm welches es f�r Linux und Windows gibt.

Die Hardware kann als Bausatz inklusiver Platine in unserem Online Shop [1] bestellt werden. Wenn man den Adapter in Eigenregie bauen will, kann man sich auch alle Pl�ne im Download Bereich[2] herunterladen (Schaltpl�ne, Platinenlayouts usw...).

Der Adapter soll nicht nur f�r AVR Controller sein, sondern es soll mit der Zeit ein breites Spektrum aufgebaut werden. 
Wenn es eine neue Software gibt kann man sie dann ganz einfach, ohne grossen Aufwand einspielen. 
Da die Quelltexte alle offen sind findet sich vielleicht der eine oder andere der mit entwickeln will. 

\subsection{Firmware Versionen}

\subsubsection{AVRISP mkII Klon}
	Dieser Klon verh�lt sich 1:1 wie ein AVRISP mkII. Bisher wurde
	er erfolgreich mit AVR Studio 4 und avrdude getestet.

\subsubsection{usbprog - AVR Porgrammierer und RS232 Schnittstelle �ber USB}
	Mit dieser kombination arbeiten wohl die meisten Entwickler.
	Normalerweise sieht es am Tisch des Entwicklers dann so aus,
	das ein Parallelport-Kabel und ein Serielles Kabel vom PC weggeht. Dazu
	wird dann meistens noch f�r das Zielboard eine Versorgungsspannung
	ben�titg. Mit dieser Firmwareversion spart man sich zwei von den
	drei Kabeln.

	Die serielle Schnittstelle wird aktuell nur von Linux (als /dev/ttyUSBx) unterst�tzt.

\subsubsection{usbprog - JTAG Interface, AVR On-Chip-Debugger}
	Diese Version ist gerade noch in der Entwicklung. Man kann dann ohne extra Hardwareaufwand
	einfach Atmels AVR Controller debuggen. 


\subsection{Hardware}


Das Herz des Adapters bilden ein ATMega32 der verbunden ist mit einem USBN9604. Alle
Schaltpl�ne und Quelltexte zu dieser sind im Downloadbereich herunterladbar[2].


Eine Anleitung zum Aufbauen der Hardware gibt es hier[3].


\subsection{Firmware installieren}

Wenn man auf einen USB Adapter umsteigt bringt es einem nichts,
wenn man dann wieder ein Parallelport- oder Serielles-Kabel
zum austauschen der Firmeware auf dem USB Adapter ben�tigt.

Daher wurde ein Mechanismus eingebaut, so dass man die Firmware
ohne extra Verbindung jederzeit austauschen kann.

Das einzige mal wann ein herk�mmliches Kabel ben�tig wird,
ist der Zeitpunkt wenn der das usbprog Hintergrundprogramm
usbprog_base.hex aufgespielt wird.

Sobald dies einmal gemacht worden ist kann man mit usbprog Online
bequem sich eine passende Firmware aussuchen und aufspielen.

SCREENSHOT


Eine genaue Anleitung wie Sie das erste Mal usbprog_base.hex aufspielen
und aktivieren finden Sie hier[4].


\subsection{Unterst�zte Software}
  \subsubsection{AVR Studio 4}

	  \subsubsection{avrdude}

		  \subsubsection{avarice}


\end{document}
